% !TEX root = ../main.tex


\chapter{Petri Netze}
Petrinetze sind Bipartiete gerichtete Grafen mit Kanten.
D.h. sie bestehen aus zwei Disjunkten Knotenmengen -- \textbf{Stellen} und \textbf{Transitionen} -- die nur mit der jeweils anderen Menge durch Kanten verbunden sind.

Die Stellen (oder auch Plätze) können mit einer beliebigen Menge an \textbf{Marken} belegt sein.
Die Menge der Marken auf allen Stellen repräsentiert den aktuellen Zustand des Systems und wird \textbf{Markierung} genannt.

Transitionen repräsentieren die möglichen \textbf{Aktionen} des Systems. 
Eine Transition ist \textbf{aktiviert}, wenn auf allen Stellen die mit eingehenden kanten verbunden sind, mindestens eine Marke liegt (bzw. eine Menge entsprechend der Kantengewichte).
Eine \textbf{aktivierte} Transition kann zu einem beliebigen Zeitpunkt \textbf{feuern}.

Feuert eine Transition werden auf allen Stellen, die mit eingehenden Kanten verbunden sind, Marken \textbf{konsumiert} und auf allen Stellen die mit einer Ausgehenden Kante verbunden sind werden Marken \textbf{produziert}.
Auch hier ist die Menge der konsumierten und produzierten Marken gleich des Kantengewichts der dazugehörigen Kante.

Somit kann ein Petrinetz als Sechs-Tupel [S, T, F, W, m\textsubscript{0}] definiert werden mit:
\begin{itemize}
    \item S - Menge der Stellen
    \item T - Menge der Transitionen
    \item F $\subseteq \{ S \times T \} \cup \{ T \times S \}$ - Menge der Kanten
    \item W: $ F \to \mathbb{N} \setminus \{0\}$ - Menge der Kantengewichte
    \item m\textsubscript{0}: $S \to \mathbb{N} \cup \{0\}$ - Anfangsmarkierung der Stellen
\end{itemize}
Auf dieser Definition aufbauend können wie erwähnt einige wichtige Eigenschaften hergeleitet werden.
Die Details sind außerhalb des Umfangs dieser Arbeit, deswegen beschränken wir uns auf eine Informale Beschreibung der drei Wichtigsten.

\section{Linearität des Schaltens}
Petrinetze lassen sich als lineare Gleichungssysteme darstellen. 
Dadurch können die Regeln der linearen Algebra auf das System anwenden. 

Mittels der \textbf{Zustandsgleichung} kann durch einfache Lösung des Gleichungssystems z.B. bereits eine Aussage darüber getroffen werden welche Markierung \texttt{nicht} von der Anfangsmarkierung aus erreicht werden können.

Außerdem lassen sich durch Stellen- und Transitionsinvarianten Aussagen über die Lebendigkeit und Beschränktheit des Systems treffen.
Ob ein System Deadlocks hat ist z.B. eng mit der Lebendigkeit des Netzes verbunden.

\section{Monotonie des Schaltens}
Im Gegensatz zu anderen Formalismen 


monotonie -> Ein weg den man in einer markierung gehen kann, kann man auch in einer größeren markierung gehen\\
localität Aktionen betreffen jeweils nur einen Teil des Zustandsraumes “Lokalitätsprinzip" C.A.Petri(Diss.1962)\\
-> kann man beim explizieten modellchecking ausnutzen\\