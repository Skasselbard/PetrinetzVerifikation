% !TEX root = ../main.tex
\chapter{Aktueller Stand}

Die Eigenschaften und Techniken die in dieser Arbeit beschrieben wurden sind nur ein kleiner Ausschnitt des Möglichen. 
Gleichzeitig ist die unterliegende Theorie immernoch nicht vollständig ausgereizt.

So ist es nicht überraschend das in diesem Gebiet weiterhin geforscht wird. 
Der zu durchsuchende Zustandsraum kann nie klein genug sein.

Ein Bestreben die benutzten Techniken zu optimieren zeigt sich z.B bei der Partial-Order-Reduction\cite{bonneland2018start} oder der Reduktion durch Symmetrie\cite{bourdil2016symmetry}.
Und während die Auswertung von Datenstrukturen in Petri-Netzen eher schwierig ist, gibt es auch hier Ansätze diese Probleme in den Griff zu bekommen\cite{xiang2017detecting}.

Doch auch die besten Reduktionstechniken nützen nichts, solange sie nicht komfortabel zu nutzen sind.
Eines der jüngeren Tools in diesem Bereich ist Charlie\cite{heiner2015charlie}. Aber auch ältere Tools werden weiter gepflegt und entwickelt, wie sich an LoLA2 zeigt\cite{wolf2018petri}.
