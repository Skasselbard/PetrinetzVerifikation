% !TEX root = ../main.tex
\chapter{Einleitung}
\label{introduction}
Eines der ältesten und verbreitetsten Modelle zur Beschreibung von, sowohl sequentiellen, als auch parallelen Rechnern, ist der endliche Automat.

Automaten eigenen sich um Aussagen über bestimmte Eigenschaften der beschriebenen Systeme zu treffen.
Eine erschöpfende Analyse wird jedoch bei größeren Modellen immer schwieriger, da die Anzahl der möglichen Zustände die angenommen werden können, gewöhnlich exponentiell mit der Größe des Systems wächst.
Dieses Phänomen wird als Zustandsexplosion bezeichnet und ist eines der größten Hindernisse im Gebiet der Softwareverifikation.

Gerade bei verteilten Systemen ist es aber selten nötig Auswirkungen von Aktionen über den gesamten Zustandsraum zu prüfen.
Viele Aktionen haben nur einen sehr lokalen Einfluss und können nur bestimmte Teile des Zustands verändern.
Die Konstruktion des kompletten Zustandsraums ist also nicht unbedingt notwendig um bestimmte Eigenschaften zu überprüfen.

Das war auch einer der Gedanken, die Carl Adam Petri umtrieb und ihn dazu inspirierte einen anderen Formalismus zur Beschreibung von Computern zu entwickeln.
Die nach ihm benannten Petri-Netze.

Petri-Netze haben einige interessante Eigenschaften, die man in der Verifikation gut ausnutzen kann.
Bevor wir allerdings zur Verifikation mit Petri Netzen kommen, sehen wir uns im folgenden Kapitel diese Eigenschaften etwas genauer an.

% Globale Zeit (gemeinsamer Takt) ab einer gewissen Komplexität nicht adäquat