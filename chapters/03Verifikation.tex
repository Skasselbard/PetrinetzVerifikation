% !TEX root = ../main.tex
\chapter{Verifikation}
Petri Netze eignen sich gut, um verteilte Systeme zu modellieren.
Solche Systeme mit einer hohen Anzahl an parallel ausführbaren Aktionen sind allerdings besonders von dem Problem der Zustandsexplosion betroffen.

Trotzdem ist es in einigen Anwendungsfällen wichtig, dass das System gewisse Eigenschaften erfüllt.
Die Softwareverifikation ist ein Bereich der Informatik, der sich mit diesem Problem beschäftigt.

Auch Systeme, die als Petri-Netz modelliert sind lassen sich verifizieren.
Damit dies effizient ablaufen kann und z.B. die Auswirkungen der Zustandsexplosion gering gehalten werden können, gibt es einige Methoden, die die Eigenschaften, die im vorherigen Kapitel vorgestellt wurden, ausnutzen.

Einen Überblick über einige wichtige sehen wir uns in diesem Kapitel an.

\section{Endlichkeit durch den Überdeckungsgraph}
Verifikation in Form von explizitem Model Checking arbeitet auf dem Zustandsraum eines Systems.
Bei dieser Methode werden solange Zustände aufgedeckt bis eine Eigenschaft verifiziert oder falsifiziert ist... Oder bis der Model Checker aus dem Speicher läuft.

Hier ist es also wichtig die Anzahl aufgedeckter Zustände niedrig zu halten.
Eine unangenehme Eigenheit einiger Systeme ist jedoch das der Zustandsraum nicht einfach nur explodiert, sondern von vornherein unendlich ist -- ärgerlich.
Ein einfaches Beispiel wäre ein Programm, das die natürlichen Zahlen aufzählt.

Glücklicherweise bietet uns die Monotonie des Schaltens hier eine Möglichkeit, eine unendliche Menge an Zuständen, in eine endliche Menge von Speicher abzulegen.

Statt alle erreichbaren Zustände aufzulisten, können wir aufhören, wenn wir feststellen das eine Markierung, eine andere überdeckt (siehe \cite{?2.2}).
So brauchen wir nur noch Speichern, dass wir einen Zustand gefunden haben der unbeschränkt ist.

Die passende Datenstruktur hierfür heißt Überdeckungsgraph und ist immer endlich groß.

\section{Ein Blick in die Zukunft mit der Zustandsgleichung}
Für einige Eigenschaften brauchen wir allerdings überhaupt keine Zustände aufzudecken.
Insbesondere für die Erreichbarkeit von Zuständen können wir die Linearität des Schaltens \cite{2.1} für unsere Zwecke nutzen.

Mit hilfe der linearen Algebra und der Zustandsgleichung können wir nur durch Lösen eines linearen Gleichungssystems erkennen, ob ein Zustand unerreichbar ist.
Das erfolgreiche Lösen des Gleichungssystems ist zwar nur eine notwendige Bedingung zur Erreichbarkeit.
Aber durch Nutzung von meist sehr effizienten Mathematik-Algorithmen, ist es oft Sinnvoll, zunächst herauszufinden ob überhaupt die Möglichkeit besteht einen Zustand zu erreichen. Danach kann man immer noch den vergleichsweise großen Zustandsraum aufdecken.

\section{Zustandsamnesie mit Sweepline}
Die Linearität des Schaltens kommt uns in einer weiteren Methode zu Gute.
Wieder um den Zustandsraum zu reduzieren. Bzw. um die benötigten Ressourcen zum Speichern der Zustände zu reduzieren.

Durch einen Fortschrittswert können wir Zustände untereinander in Beziehung setzen.
Oft sind neue Zustände nicht mehr direkt abhängig von älteren zuvor aufgedeckten.
Wenn wir zusätzlich feststellen, dass es keine Möglichkeit mehr gibt, diese vergangenen Zustände zu erreichen, können wir diese auch ohne schlechtes Gewissen vergessen.

Die frei gewordenen Ressourcen können so für die weitere Suche recycelt werden.

\section{Symmetrie -- nicht nur eine Frage der Optik}
Die Graphentheorie und die Lokalität des Schaltens\cite{2.3} unterstützen uns um ein weiteres Mal den Zustandsraum, den wir aufdecken müssen, zu reduzieren.

Da Petri-Netze selbst als Graph aufgefasst werden können, können auch Graphenalgorithmen  und Konzepte auf sie angewendet werden. 
Besonders nützlich sind hier die Graphautomorphismen bzw. die Symmetrie.

Teile des Petri-Netzes können sich zu anderen Teilen symmetrisch verhalten. 
D.h. wenn ein Teil des Petri-Netzes bestimmte Aktionen ausführen kann, kann es auch ein anderer Teil der sich zum ersten symmetrisch verhält.

Solche Symmetrien lassen sich besonders in verteilten Systemen beobachten, die das gleiche Programm auf verschiedenen (oder allen) Knoten ausführen.
Hier kann offensichtlich jeder Knoten exakt die gleichen Aktionen ausführen wie alle anderen.

Standardmäßig kann ein Model Checker diese Information jedoch nicht nutzen und müsste trotzdem das gesamte System überprüfen.
Durch das Erkennen von symmetrischen Strukturen im Petri-Netz können wir unsere Rechenleistung jedoch für weniger repetitive Aufgaben nutzen und neben Zeit wieder Speicherplatz einsparen.

\section{Effizienz durch Ordnung mit Partial-Order-Reduction}
Partial-Order-Reduction ist eine weitere Technik um den Zustandsraum durch die Lokalitätseigenschat zu reduzieren. 
Und wieder wird die Unabhängigkeit von Aktionen ausgenutzt, die besonders durch Parallele Systeme eingeführt wird.

Mit Hilfe von Stubborn-Sets können Zustandsfolgen gefunden werden, die die gleichen Aktionen (in unterschiedlicher Reihenfolge) nutzen und denselben Endzustand erreichen.
Hier ist es im Sinne der Verifikation wieder egal, welche Reihenfolge gewählt wird.
Das Schalten eines Pfads ist genug um (passende) Eigenschaften zu untersuchen.

\section{Das Ganze ist mehr als die Summe seiner Teile}
Die meisten der zuvor genannten Techniken, haben das Ziel die Menge der Zustände zu reduzieren und dieses immer wieder erwähnte Problem der Zustandsexplosion in den Griff zu bekommen.
Die einzelnen Techniken sind allerdings nicht immer auf jedes Problem anwendbar. Oft erhalten sie nur Teile der Netzeigenschaften.

Angenehmer weise lassen sich die Techniken untereinander Kombinieren um die Menge der zu untersuchenden Zustände auf ein Minimum zu reduzieren.

Viele Probleme der Praxis lassen sich so auch ohne Kühlschrankgroße spezial Rechner lösen.
Soweit, dass Systeme oft interaktiv zu überprüfen sind.